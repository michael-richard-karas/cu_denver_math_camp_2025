\documentclass[aspectratio=169]{beamer}
\usepackage{booktabs}
\usepackage{graphicx}
\usepackage{adjustbox}
% \documentclass[aspectratio=43]{beamer}

% Define `accent`/`accent2` colors for theme customization
\definecolor{accent}{HTML}{006896}
\definecolor{accent2}{HTML}{E64173}

% Beamer theme
% Define colors ----------------------------------------------------------------
\usepackage{xcolor}

\definecolor{purple}{HTML}{695693}
\definecolor{cranberry}{HTML}{E64173}
\definecolor{orange}{HTML}{D65616}
\definecolor{navy}{HTML}{006896}
\definecolor{teal}{HTML}{1A505A}
\definecolor{ruby}{HTML}{9a2515}
\definecolor{alice}{HTML}{107895}
\definecolor{daisy}{HTML}{EBC944}
\definecolor{coral}{HTML}{F26D21}
\definecolor{kelly}{HTML}{829356}
\definecolor{slate900}{HTML}{131516}
\definecolor{asher}{HTML}{555F61}
\definecolor{slate}{HTML}{314F4F}

% Slate from Tailwind Colors
\definecolor{slate50}{HTML}{f8fafc}
\definecolor{slate100}{HTML}{f1f5f9}
\definecolor{slate200}{HTML}{e2e8f0}
\definecolor{slate300}{HTML}{cbd5e1}
\definecolor{slate400}{HTML}{94a3b8}
\definecolor{slate500}{HTML}{64748b}
\definecolor{slate600}{HTML}{475569}
\definecolor{slate700}{HTML}{334155}
\definecolor{slate800}{HTML}{1e293b}
\definecolor{slate900}{HTML}{0f172a}
\definecolor{slate950}{HTML}{020617}

% Easily color text
\newcommand\purple[1]{{\color{purple}#1}}
\newcommand\cranberry[1]{{\color{cranberry}#1}}
\newcommand\orange[1]{{\color{orange}#1}}
\newcommand\navy[1]{{\color{navy}#1}}
\newcommand\teal[1]{{\color{teal}#1}}
\newcommand\kelly[1]{{\color{kelly}#1}}
\newcommand\ruby[1]{{\color{ruby}#1}}
\newcommand\alice[1]{{\color{alice}#1}}
\newcommand\daisy[1]{{\color{daisy}#1}}
\newcommand\coral[1]{{\color{coral}#1}}

% Color background of text
\newcommand\bgNavy[1]{{\colorbox{navy!80!white}{#1}}}
\newcommand\bgOrange[1]{{\colorbox{orange!80!white}{#1}}}
\newcommand\bgTeal[1]{{\colorbox{teal!80!white}{#1}}}
\newcommand\bgPurple[1]{{\colorbox{purple!80!white}{#1}}}
\newcommand\bgKelly[1]{{\colorbox{kelly!80!white}{#1}}}
\newcommand\bgRuby[1]{{\colorbox{ruby!80!white}{#1}}}
\newcommand\bgAlice[1]{{\colorbox{alice!80!white}{#1}}}
\newcommand\bgDaisy[1]{{\colorbox{daisy!80!white}{#1}}}
\newcommand\bgCoral[1]{{\colorbox{coral!80!white}{#1}}}
\newcommand\bgCranberry[1]{{\colorbox{cranberry!80!white}{#1}}}

% Beamer Options ---------------------------------------------------------------

% Background
\setbeamercolor{background canvas}{bg = white}

% Change text margins
\setbeamersize{text margin left = 15pt, text margin right = 15pt} 

% \alert
\setbeamercolor{alerted text}{fg = accent2}

% Frame title
\setbeamercolor{frametitle}{bg = white, fg = slate900}
\setbeamercolor{framesubtitle}{bg = white, fg = accent}
\setbeamerfont{framesubtitle}{size = \small, shape = \itshape}

% Page numbering
\usepackage{appendixnumberbeamer}
\setbeamercolor{page number in head/foot}{fg=slate600}
\setbeamertemplate{footline}[frame number]

% Table of Contents
\setbeamercolor{section in toc}{fg = slate700}
\setbeamercolor{subsection in toc}{fg = slate900}

% Button 
\setbeamercolor{button}{bg = white, fg = slate900}
\setbeamerfont{button}{}
\setbeamercolor{button border}{fg = accent}

% Remove navigation symbols
\setbeamertemplate{navigation symbols}{}

% Table and Figure captions
\setbeamercolor{caption}{fg = slate900!70!white}
\setbeamercolor{caption name}{fg=slate900}
\setbeamerfont{caption name}{shape = \itshape}

% Links
\usepackage{hyperref}
\hypersetup{
  colorlinks = true,
  linkcolor = accent2,
  filecolor = accent2,
  urlcolor = accent2,
  citecolor = accent2,
}

% Line spacing
\usepackage{setspace}
\setstretch{1.3}

% Remove annoying over-full box warnings
\vfuzz2pt 
\hfuzz2pt


% Title page -------------------------------------------------------------------
\setbeamercolor{title}{fg = slate900}
\setbeamercolor{subtitle}{fg = accent}

%% Custom \maketitle and \titlepage
\setbeamertemplate{title page}
{
  %\begin{centering}
  \vspace{20mm}
  {\Large \usebeamerfont{title}\usebeamercolor[fg]{title}\inserttitle}\\ \vskip0.25em%
  \ifx\insertsubtitle\@empty%
  \else%
    {\usebeamerfont{subtitle}\usebeamercolor[fg]{subtitle}\insertsubtitle\par}%
  \fi% 
  {\vspace{10mm}\insertauthor}\\
  {\color{asher}\small{\insertdate}}\\
  %\end{centering}
}

% Table of Contents with Sections ----------------------------------------------
\setbeamerfont{myTOC}{series=\bfseries, size=\Large}
\AtBeginSection[]{
  \begin{frame}{Roadmap}
    \tableofcontents[current]   
  \end{frame}
}

% Block ------------------------------------------------------------------------
\usepackage{tcolorbox}

\defbeamertemplate{block begin}{framed}[1][] {
  \begin{tcolorbox}[colback=slate50, colframe=slate200, arc=0mm]
  {
    \vskip\smallskipamount%
    \ifthenelse{\equal{#1}{}}{}{%
      \raggedright\usebeamerfont*{block title}\usebeamercolor[fg]{title}%
      \textbf{\insertblocktitle}%
      \vskip\medskipamount%
    }
  }%
  \raggedright%
  \usebeamerfont{block body}%
}
\defbeamertemplate{block end}{framed}[1][] {
  \vskip\smallskipamount\end{tcolorbox}
}
\setbeamertemplate{blocks}[framed]

% Colors from plugging base color into https://uicolors.app/create
\usepackage{ifthen}
\newenvironment*{slateBlock}[1]{%
  \begin{tcolorbox}[colback=slate50, colframe=slate200, arc=0mm]{
    \vskip\smallskipamount%
    \ifthenelse{\equal{#1}{}}{}{%
      \raggedright\usebeamerfont*{block title}\usebeamercolor[fg]{title}%
      \textbf{#1}%
      \vskip\medskipamount%
    }%
  }%
  \raggedright%
  \usebeamerfont{block body}%
}{%
  \vskip\smallskipamount\end{tcolorbox}
}

\definecolor{purple50}{HTML}{f9f8fc}
\definecolor{purple100}{HTML}{f1eff8}
\definecolor{purple200}{HTML}{e6e2f2}
\newenvironment*{purpleBlock}[1]{%
  \begin{tcolorbox}[colback=purple50, colframe=purple200, arc=0mm]{
    \vskip\smallskipamount%
    \ifthenelse{\equal{#1}{}}{}{%
      \raggedright\usebeamerfont*{block title}\usebeamercolor[fg]{title}%
      \textbf{#1}%
      \vskip\medskipamount%
    }%
  }%
  \raggedright%
  \usebeamerfont{block body}%
}{%
  \vskip\smallskipamount\end{tcolorbox}
}

\definecolor{cranberry50}{HTML}{fdf2f6}
\definecolor{cranberry100}{HTML}{fbe8ef}
\definecolor{cranberry200}{HTML}{fad0e0}
\newenvironment*{cranberryBlock}[1]{%
  \begin{tcolorbox}[colback=cranberry50, colframe=cranberry200, arc=0mm]{
    \vskip\smallskipamount%
    \raggedright\usebeamerfont*{block title}\usebeamercolor[fg]{title}%
    \textbf{#1}%
  }%
  \vskip\medskipamount%
  \raggedright%
  \usebeamerfont{block body}%
}{%
  \vskip\smallskipamount\end{tcolorbox}
}

% Bullet points ----------------------------------------------------------------

%% Fix left-margins
\settowidth{\leftmargini}{\usebeamertemplate{itemize item}}
\addtolength{\leftmargini}{\labelsep}

%% enumerate item color
\setbeamercolor{enumerate item}{fg = slate600}
\setbeamerfont{enumerate item}{size = \small}
\setbeamertemplate{enumerate item}{\insertenumlabel.}

%% itemize
\setbeamercolor{itemize item}{fg = slate600}
\setbeamerfont{itemize item}{size = \small}
\setbeamertemplate{itemize item}[circle]

%% right arrow for subitems
\setbeamercolor{itemize subitem}{fg = slate600}
\setbeamerfont{itemize subitem}{size = \small}
\setbeamertemplate{itemize subitem}{$\rightarrow$}

\setbeamercolor{itemize subsubitem}{fg = slate600}
\setbeamerfont{itemize subsubitem}{size = \small}
\setbeamertemplate{itemize subsubitem}[square]

% References -------------------------------------------------------------------

%% Bibliography Font, roughly matching aea
\setbeamerfont{bibliography item}{size = \footnotesize}
\setbeamerfont{bibliography entry author}{size = \footnotesize, series = \bfseries}
\setbeamerfont{bibliography entry title}{size = \footnotesize}
\setbeamerfont{bibliography entry location}{size = \footnotesize, shape = \itshape}
\setbeamerfont{bibliography entry note}{size = \footnotesize}

\setbeamercolor{bibliography item}{fg = slate900}
\setbeamercolor{bibliography entry author}{fg = accent!60!slate900}
\setbeamercolor{bibliography entry title}{fg = slate900}
\setbeamercolor{bibliography entry location}{fg = slate900}
\setbeamercolor{bibliography entry note}{fg = slate900}

%% Remove bibliography symbol in slides
\setbeamertemplate{bibliography item}{}


% \begin{columns} --------------------------------------------------------------
\usepackage{multicol}


% Fonts ------------------------------------------------------------------------
% Beamer Option to use custom fonts
\usefonttheme{professionalfonts}

% \usepackage[utopia, smallerops, varg]{newtxmath}
% \usepackage{utopia}
\usepackage[sfdefault,light]{roboto}

% Small adjustments to text kerning
\usepackage{microtype}

% For \underbracket/\overbracket
\usepackage{mathtools}

% References -------------------------------------------------------------------
\usepackage[
    citestyle= authoryear,
    style = authoryear,
    natbib = true, 
    backend = biber
]{biblatex}

% Smaller font-size for references
\renewcommand*{\bibfont}{\small}

% Remove "In:"
\renewbibmacro{in:}{}

% Color citations for slides 
\newenvironment{citecolor}
    {\footnotesize\begin{color}{accent2}}
    {\end{color}}

\newcommand{\citetcolor}[1]{{\footnotesize\textcolor{gray}{\citet{#1}}}}
\newcommand{\citepcolor}[1]{{\footnotesize\textcolor{gray}{\citep{#1}}}}

% Tables -----------------------------------------------------------------------

% When tables are too big, use adjustbox
% \begin{adjustbox}{width = 1.2\textwidth, center}
\usepackage{adjustbox}
\usepackage{array}
\usepackage{threeparttable, booktabs, adjustbox}
    
% Fix \input with tables
% \input fails when \\ is at end of external .tex file
\makeatletter
\let\input\@@input
\makeatother

% Tables too narrow
% \begin{tabularx}{\linewidth}{cols}
% col-types: X - center, L - left, R -right
% Relative scale: >{\hsize=.8\hsize}X/L/R
\usepackage{tabularx}
\newcolumntype{L}{>{\raggedright\arraybackslash}X}
\newcolumntype{R}{>{\raggedleft\arraybackslash}X}
\newcolumntype{C}{>{\centering\arraybackslash}X}

% Table Highlighting -----------------------------------------------------------
% Create top-left and bottom-right markets in tabular cells with a unique matching id and these commands will outline those cells
\usepackage[beamer,customcolors]{hf-tikz}
\usetikzlibrary{calc}
\usetikzlibrary{fit,shapes.misc}

% To set the hypothesis highlighting boxes red.
\newcommand\marktopleft[1]{%
    \tikz[overlay,remember picture] 
        \node (marker-#1-a) at (0,1.5ex) {};%
}
\newcommand\markbottomright[1]{%
    \tikz[overlay,remember picture] 
        \node (marker-#1-b) at (0,0) {};%
    \tikz[accent!80!slate900, ultra thick, overlay, remember picture, inner sep=4pt]
        \node[draw, rectangle, fit=(marker-#1-a.center) (marker-#1-b.center)] {};%
}

% Figures ----------------------------------------------------------------------

% \imageframe{img_name}
% from https://github.com/mattslate900well/cousteau
\newcommand{\imageframe}[1]{%
    \begin{frame}[plain]
        \begin{tikzpicture}[remember picture, overlay]
            \node[at = (current page.center), xshift = 0cm] (cover) {%
                \includegraphics[keepaspectratio, width=\paperwidth, height=\paperheight]{#1}
            };
        \end{tikzpicture}
    \end{frame}%
}

% subfigures
\usepackage{subfigure}

% Highlight slide --------------------------------------------------------------
% \begin{transitionframe} Text \end{transitionframe}
% from paulgp's beamer tips
\newenvironment{transitionframe}{
    \setbeamercolor{background canvas}{bg=accent!60!black}
    \begin{frame}\color{white}\LARGE\centering
}{
    \end{frame}
}



% Title --------------------------------------------
\title{CU Denver Math Camp - Limits \& Derivatives}
\subtitle{Additional Topics}
\date{August 5, 2024}
\author{Michael R. Karas}

\begin{document}

% ------------------------------------------------------------------------------
\begin{frame}
\maketitle

% \vspace{2.5mm}
{\footnotesize University of Colorado, Boulder}
\end{frame}
% ------------------------------------------------------------------------------

%------------------------------------------------------------------------------
\begin{frame}{Additional Topics}\label{main1}

\begin{itemize}
	\begin{itemize}
		\item Integral Calculus	
		\item Sets and Numbers
	\end{itemize}
\end{itemize}
\end{frame}
%------------------------------------------------------------------------------

%------------------------------------------------------------------------------
\begin{frame}{Integral Calculus}\label{main1}
	An antiderivative of a function $f(x)$ is a function $F(x)$ whose derivative is the original $F:F' = f$
\begin{itemize}
	\begin{itemize}
		\item The function $F$ is also called the indefinite integral of $f$, $F(x) = \int f(x) dx$
	\end{itemize}
\end{itemize}
\end{frame}
%------------------------------------------------------------------------------

%------------------------------------------------------------------------------
\begin{frame}{Integral Rules}\label{main1}
\begin{itemize}
	\begin{itemize}
		\item $\int k \, dx = kx + C \quad $ where $k$ is a constant
		\item $\int x^n \, dx = \frac{x^{n+1}}{n+1} + C$ for $n \ne -1$
		\item $ \int a \cdot f(x)\, dx = a \int f(x)\, dx $
		\item $ \int [f(x) \pm g(x)]\, dx = \int f(x)\, dx \pm \int g(x)\, dx $
		\item $ \int \frac{1}{x} \, dx = \ln|x| + C $
		\item $ \int e^x \, dx = e^x + C $
		\item Integration by parts
		\begin{itemize}
			\item $ \int u(x) v'(x) \, dv = u(x) v(x) - \int u'(x) v(x) \, du $
		\end{itemize}
	\end{itemize}
\end{itemize}
\end{frame}
%------------------------------------------------------------------------------

%------------------------------------------------------------------------------
\begin{frame}{Indefinite Integral Example}
Evaluate the following:
\[
\int (3x^2 + 2x + 1) \, dx
\]

Apply the addition rule:
\[
\int 3x^2 \, dx + \int 2x \, dx + \int 1 \, dx
\]

Apply the power rule:
\[
\int x^n \, dx = \frac{x^{n+1}}{n+1} + C
\]

Integrate each term:
\[
3 \cdot \frac{x^3}{3} + 2 \cdot \frac{x^2}{2} + x + C
\]
\[
x^3 + x^2 + x + C
\]
\end{frame}
%------------------------------------------------------------------------------

%------------------------------------------------------------------------------
\begin{frame}{Fundamental Theorem of Calculus}\label{main1}
	For numbers $a$ and $b$, the definite integral of $f(x)$ from $a$ to $b$ is $F(b) - F(a)$, where $F(x)$ is an antiderivative of $f$.
\begin{itemize}
	\begin{itemize}
		\item $\int_a^b f(x) \, dx = F(b) - F(a)$ where $F' = f$
	\end{itemize}
	\item The Fundamental Theorem of Calculus states that if we iterate each time dividing $[a,b]$ into smaller and smaller sub-intervals, in the limit we obtain the definite integral $\int_a^b f(x) \, dx$
		\begin{itemize}
		\item $\lim_{\Delta \to 0} \sum_{i=1}^{N} f(x_{i})\Delta = \int_a^b f(x) \, dx$
	\end{itemize}
\end{itemize}
\end{frame}
%------------------------------------------------------------------------------

%------------------------------------------------------------------------------
\begin{frame}{Sets}\label{main1}
	A set is any well-specified collection of elements
\begin{itemize}
	\begin{itemize}
		\item For any set $A$, we write $a \in A$ to indicate $a$ is a member of set $A$, and $a \notin A$ to indicate that $a$ is not in the set $A$
		\item A set which contains no elements is called the empty set or null set and is denoted by $\varnothing$
		\item Example of standard notation for sets: the set of all non-negative numbers is written as
		\begin{itemize}
			\item $ R_{+} \equiv \{x \in R : x \geq 0 \} $
		\end{itemize}
	\end{itemize}
\end{itemize}
\end{frame}
%------------------------------------------------------------------------------

%------------------------------------------------------------------------------
\begin{frame}{Operations with Sets}\label{main1}
\begin{itemize}
	\begin{itemize}
		\item $A \cup B$, spoken "$A$ union $B$," is the set of all elements that are either in $A$ or in $B$ (or in both)
		\begin{itemize}
			\item $A \cup B \equiv \{ x : x \in A $ or $ x \in B \} $
		\end{itemize}
	\item $A \cap B$, spoken "$A$ intersect $B$," is the set of all elements that are common to both $A$ and $B$
		\begin{itemize}
			\item $A \cap B \equiv \{ x : x \in A $ and $ x \in B \} $
		\end{itemize}
	\item $A - B$, or sometimes $A \setminus B$, spoken "$A$ minus $B$," is the set of all elements of $A$ that are not in $B$
		\begin{itemize}
			\item $A - B \equiv \{ x : x \in A $ and $ x \notin B \} $
		\end{itemize}
		\end{itemize}
	\end{itemize}
\end{frame}
%------------------------------------------------------------------------------

%------------------------------------------------------------------------------
\begin{frame}{Number Sets}
\begin{itemize}
    \item \textbf{Natural Numbers} (\( \mathbb{N} \)):\\
    \[
    \mathbb{N} = \{1, 2, 3, 4, \dots\}
    \]
    The set of positive whole numbers.

    \item \textbf{Integers} (\( \mathbb{Z} \)):\\
    \[
    \mathbb{Z} = \{\dots, -3, -2, -1, 0, 1, 2, 3, \dots\}
    \]
    The set of whole numbers including negative numbers and zero.
\end{itemize}
\end{frame}
%------------------------------------------------------------------------------

%------------------------------------------------------------------------------
\begin{frame}{Number Sets}
\begin{itemize}
    \item \textbf{Rational Numbers} (\( \mathbb{Q} \)):\\
    \[
    \mathbb{Q} = \left\{ \frac{a}{b} \mid a, b \in \mathbb{Z}, b \neq 0 \right\}
    \]
    Numbers that can be expressed as a fraction of two integers.

    \item \textbf{Real Numbers} (\( \mathbb{R} \)):\\
    \[
    \mathbb{R} = \text{All points on the number line, including rationals and irrationals}
    \]
    Includes rational and irrational numbers such as \( \pi \), \( \sqrt{2} \).

\end{itemize}
\end{frame}
%------------------------------------------------------------------------------


\end{document}
