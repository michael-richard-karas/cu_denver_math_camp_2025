\documentclass[aspectratio=169]{beamer}
\usepackage{booktabs}
\usepackage{graphicx}
\usepackage{adjustbox}
\usepackage{amsmath}
% \documentclass[aspectratio=43]{beamer}

% Define `accent`/`accent2` colors for theme customization
\definecolor{accent}{HTML}{006896}
\definecolor{accent2}{HTML}{E64173}

% Beamer theme
\input{preamble.tex}

% Title --------------------------------------------
\title{CU Denver Math Camp - Limits \& Derivatives}
\subtitle{Day 1}
\date{August 4, 2025}
\author{Michael R. Karas}

\begin{document}

% ------------------------------------------------------------------------------
\begin{frame}
\maketitle

% \vspace{2.5mm}
{\footnotesize University of Colorado, Boulder}
\end{frame}
% ------------------------------------------------------------------------------

%------------------------------------------------------------------------------
\begin{frame}{About Me}\label{main1}
Michael Karas, PhD Candidate at CU Boulder
\begin{itemize}
	\begin{itemize}
		\item michael.karas@colorado.edu
		\item Entering my 5th year, will be on job market Fall 2025
	\end{itemize}
	\item Environmental Economics, Economic History, Industrial Organization, Applied Micro
	\begin{itemize}
		\item Development of the Gas Utility Industry
			\begin{itemize}
				\item The initial adoption of public utility commissions
				\item Role of incumbent firms in the transition from manufactured gas to natural gas
				\item The substitutability and complementarity between gas and electricity utilities
				\item The development of block pricing tariffs in the industry
			\end{itemize}
	\end{itemize}
	\item Teaching: Intermediate Microeconomics, Principles of Macroeconomics
	\item Hobbies: Cycling, Skiing
\end{itemize}

\end{frame}
%------------------------------------------------------------------------------


%------------------------------------------------------------------------------
\begin{frame}{Getting to Know Each Other}\label{main1}
\begin{itemize}
	\item{Name}
	\item{Hometown}
	\item{Interests in Economics}
	\item{Hobbies or something about yourself}
\end{itemize}

\end{frame}
%------------------------------------------------------------------------------

%------------------------------------------------------------------------------
\begin{frame}{Grad School Advice}\label{main1}
\begin{itemize}
	\item It will be challenging
	\begin{itemize}
		\item Everyone struggles, even the ``top" students
		\item It may take a while to master the material, probably won't grasp everything on the first attempt
	\item What matters is determination, discipline, and consistency: put the time in and the results will come
	\item Treat it like a job
\end{itemize}
	\item Study Tips
	\begin{itemize}
	\item Practice makes perfect: get old homework/exam from upper students
	\item Study with each other
	\item Recognize when diminishing marginal returns start to set in and take a break
	\end{itemize}
	\item Don't neglect your mental health.  It's okay to take breaks.
	\begin{itemize}
		\item Have fun! CO is a great place, develop a life outside of school
	\end{itemize}
	\end{itemize}
\end{itemize}

\end{frame}
%------------------------------------------------------------------------------

%------------------------------------------------------------------------------
\begin{frame}{Day 1 Topics}\label{main1}
\begin{itemize}
	\begin{itemize}
		\item Limits
		\item Limit Rules
		\item Derivative Definition
		\item Derivative Rules
		\item Natural Log and Exponent Rules
		\item Higher-order Derivatives
	\end{itemize}
\end{itemize}

\end{frame}
%------------------------------------------------------------------------------

%------------------------------------------------------------------------------
\begin{frame}{Limits}\label{main1}
	The limit of \( f(x) \) as \( x \) approaches \( a \) is written as:
\[
\lim_{x \to a} f(x)
\]
\begin{itemize}
\begin{itemize}
	\item The behavior of the function as its input approaches \( a \)
	\item In certain cases, can be evaluated by plugging in \( x = a \)
	\item \( \lim_{x \to a} c = c \) for a constant \( c \)
	\item \(\lim_{x \to a} x = a \) for a constant \( c \)
\end{itemize}
\end{itemize}

\end{frame}
%------------------------------------------------------------------------------

%------------------------------------------------------------------------------
\begin{frame}{Limits}\label{main1}
The limit from the left and right are
\begin{itemize}
\begin{itemize}
    \item \(\lim_{x \to a^-} f(x) = A\) “as \(x\) approaches \(a\) from the left (\(-\infty\))”
    \item \(\lim_{x \to a^+} f(x) = A\) “as \(x\) approaches \(a\) from the right (\(\infty\))”
\end{itemize}
\end{itemize}
If \(\lim_{x \to a^-} f(x) = A = \lim_{x \to a^+} f(x)\) then the limit exists and we write
\[
\lim_{x \to a} f(x) = A
\]
\begin{itemize}
    \item Note: the limit does not exist if \(A = \pm\infty\), even if the left and right limit both equal \(\pm\infty\)
\end{itemize}
\end{frame}
%------------------------------------------------------------------------------

%------------------------------------------------------------------------------
\begin{frame}{Limits}\label{main1}
    \begin{figure}
        \centering
        \includegraphics[width=0.6\textwidth]{/Users/michaelkaras/Library/CloudStorage/OneDrive-UCB-O365/class_materials/cu_denver_math_camp_2024/limits_example.jpg}
    \end{figure}
\end{frame}
%------------------------------------------------------------------------------

%------------------------------------------------------------------------------
\begin{frame}{Limit Rules}\label{main1}
    \[
    \lim_{x \to a} f(x) = L \quad \lim_{x \to a} g(x) = M
    \]
    \vspace{1em}
    \begin{itemize}
        \item \textbf{Constant Multiple Rule:} \quad $\lim_{x \to a} [a f(x)] = a L$
        \item \textbf{Sum/Difference Rule:} \quad $\lim_{x \to a} [f(x) \pm g(x)] = L \pm M$
        \item \textbf{Product Rule:} \quad $\lim_{x \to a} [f(x) \cdot g(x)] = L \cdot M$
        \item \textbf{Quotient Rule:} \quad $\lim_{x \to a} \left[ \frac{f(x)}{g(x)} \right] = \frac{L}{M}, \quad M \neq 0$
        \item \textbf{Power Rule:} \quad $\lim_{x \to a} [(f(x))^n] = L^n, \quad n > 0$
    \end{itemize}
\end{frame}
%------------------------------------------------------------------------------

%------------------------------------------------------------------------------
\begin{frame}{Limit Practice Problems}\label{main1}
    \vspace{-4cm}
    \[
    \lim_{x \to 0} (3 + 2x^{2})
    \]
\end{frame}
%------------------------------------------------------------------------------

%------------------------------------------------------------------------------
\begin{frame}{Limit Practice Problems}\label{main1}
	\vspace{-4cm}
    \[
    \lim_{x \to 1} \frac{x^{2} + 7x - 8}{x - 1} 
    \]
\end{frame}
%------------------------------------------------------------------------------

%------------------------------------------------------------------------------
\begin{frame}{Definition of a Derivative}\label{main1}
The derivative of \(f\) is defined as
\[
f'(x) = \lim_{h \to 0} \frac{f(x + h) - f(x)}{h}
\]
If this limit exists, then we say \(f\) is differentiable at \(x\).
\begin{itemize}
	\begin{itemize}
    \item Find \(f(x + h)\). Ex: \(f(x) = x^2\), \(f(x + h) = (x + h)^2 = x^2 + 2xh + h^2\)
    \item Find \(f(x + h) - f(x)\)
    \item Find \(\frac{f(x + h) - f(x)}{h}\) and simplify until \(h = 0\) doesn’t divide by 0
    \item Plug-in \(h = 0\) to determine the limit
    \end{itemize}
\end{itemize}
\end{frame}
%------------------------------------------------------------------------------

%------------------------------------------------------------------------------
\begin{frame}{Definition of a Derivative Practice Problems}\label{main1}
	\vspace{-4cm}
    \[
    f(x) = x^{2}
    \]
\end{frame}
%------------------------------------------------------------------------------

%------------------------------------------------------------------------------
\begin{frame}{Derivative Notation}\label{main1}
All of these mean “take the derivative of \(f\) with respect to \(x\)”:
\begin{itemize}
\begin{itemize}
    \item \(f'(x)\)
    \item \(\frac{df(x)}{dx}\)
    \item \(\frac{d}{dx} f(x)\)
    \item \(\frac{dy}{dx}\) if \(y = f(x)\)
    \item \(y'\) (sometimes)
    \item \( \dot{y} \) (particularly in Macro)
\end{itemize}
\end{itemize}
\end{frame}
%------------------------------------------------------------------------------

%------------------------------------------------------------------------------
\begin{frame}{Derivative Rules}\label{main1}
Let \(f(x)\) and \(g(x)\) be differentiable functions:
\begin{itemize}
    \item \textbf{Derivative of a Constant:} \(f(x) = c\), \(f'(x) = 0\)
    \item \textbf{“Power Rule”:} \(f(x) = x^n\), \(f'(x) = nx^{n-1}\) 
    \item \textbf{Sum/Difference Rule:} \quad $\frac{d}{dx} [f(x) \pm g(x)] = f'(x) \pm g'(x)$
    \item \textbf{Product Rule:} \quad $\frac{d}{dx} [f(x) \cdot g(x)] = f'(x) \cdot g(x) + f(x) \cdot g'(x)$
    \item \textbf{Quotient Rule:} \quad $\frac{d}{dx} \left( \frac{f(x)}{g(x)} \right) = \frac{f'(x) \cdot g(x) - f(x) \cdot g'(x)}{[g(x)]^2}$
    \item \textbf{Chain Rule:} \quad $\frac{d}{dx} f(g(x)) = f'(g(x)) \cdot g'(x)$
\end{itemize}
\end{frame}
%------------------------------------------------------------------------------

%------------------------------------------------------------------------------
\begin{frame}{Derivative Practice Problems}\label{main1}
	\vspace{-4cm}
    \[
    f(x) = 9x^{10}
    \]
\end{frame}
%------------------------------------------------------------------------------

%------------------------------------------------------------------------------
\begin{frame}{Derivative Practice Problems}\label{main1}
	\vspace{-4cm}
    \[
    f(x) = 8x^{4} + 2 \sqrt{x}
    \]
\end{frame}
%------------------------------------------------------------------------------

%------------------------------------------------------------------------------
\begin{frame}{Derivative Practice Problems}\label{main1}
	\vspace{-4cm}
    \[
    f(x) = \sqrt{x} * 6x^{4}
    \]
\end{frame}
%------------------------------------------------------------------------------

%------------------------------------------------------------------------------
\begin{frame}{Derivative Practice Problems}\label{main1}
	\vspace{-4cm}
    \[
    f(x) = \frac{x + 1}{x - 1}
    \]
\end{frame}
%------------------------------------------------------------------------------

%------------------------------------------------------------------------------
\begin{frame}{Derivative Practice Problems}\label{main1}
	\vspace{-4cm}
    \[
    f(x) = 5u^{4} ; u = 1 + x^{2}
    \]
\end{frame}
%------------------------------------------------------------------------------

%------------------------------------------------------------------------------
\begin{frame}{Properties of Exponents}\label{main1}
Let $a$ and $b$ be real numbers and $m$ and $n$ be integers. Then the following properties of exponents hold, provided that all of the expressions appearing in a particular equation are defined:
\begin{itemize}
\begin{itemize}
    \item $a^m a^n = a^{m+n}$
    \item $(a^m)^n = a^{mn}$
    \item $(ab)^m = a^m b^m$
    \item $\frac{a^m}{a^n} = a^{m-n}$, $a \neq 0$
    \item $\left(\frac{a}{b}\right)^m = \frac{a^m}{b^m}$, $b \neq 0$
    \item $a^{-m} = \frac{1}{a^m}$, $a \neq 0$
    \item $a^{\frac{1}{n}} = \sqrt[n]{a}$
    \item $a^0 = 1$, $a \neq 0$
    \item $a^{\frac{m}{n}} = \sqrt[n]{a^m} = \left(\sqrt[n]{a}\right)^m$
\end{itemize}
\end{itemize}
\end{frame}
%------------------------------------------------------------------------------

%------------------------------------------------------------------------------
\begin{frame}{Derivative Practice Problems}\label{main1}
	\vspace{-4cm}
    \[
    f(x) = \frac{x + 1}{x^{5}}
    \]
\end{frame}
%------------------------------------------------------------------------------

%------------------------------------------------------------------------------
\begin{frame}{Properties of Natural Log}\label{main1}
Note that logs are only defined for positive values of x:
\begin{itemize}
\begin{itemize}
    \item $\ln(xy) = \ln(x) + \ln(y)$
    \item $\ln\left(\frac{x}{y}\right) = \ln(x) - \ln(y)$
    \item $\ln(x^y) = y \cdot \ln(x)$
    \item $\ln(e^x) = x$
    \item $e^{\ln(x)} = x$
    \item $\ln(e) = 1$
    \item $\ln(1) = 0$
    \item $\ln(1) = undefined$
\end{itemize}
\end{itemize}
\end{frame}
%------------------------------------------------------------------------------

%------------------------------------------------------------------------------
\begin{frame}{Derivative Rules for Exponents and Natural Log}\label{main1}
\begin{itemize}
\begin{itemize}
    \item \(f(x) = e^x\), \(f'(x) = e^x\)
    \item \(f(x) = \ln(x)\), \(f'(x) = \frac{1}{x}\)
\end{itemize}
\end{itemize}
\end{frame}
%------------------------------------------------------------------------------

%------------------------------------------------------------------------------
\begin{frame}{Natural Log/Exponent Derivative Practice Problems}\label{main1}
	\vspace{-4cm}
    \[
    f(x) = 3x^{2}*ln(x)
    \]
\end{frame}
%------------------------------------------------------------------------------


%------------------------------------------------------------------------------
\begin{frame}{Natural Log/Exponent Derivative Practice Problems}\label{main1}
	\vspace{-4cm}
    \[
    f(x) = x^{4}*e^{x}
    \]
\end{frame}
%------------------------------------------------------------------------------

%------------------------------------------------------------------------------
\begin{frame}{Natural Log/Exponent Derivative Practice Problems}\label{main1}
	\vspace{-4cm}
    \[
    f(x) = e^{x}*(1-x)^{4}
    \]
\end{frame}
%------------------------------------------------------------------------------

%------------------------------------------------------------------------------
\begin{frame}{Higher-order Derivatives}\label{main1}
The second-order derivative is

\[
f''(x) = \frac{d}{dx} f'(x)
\]

We could keep differentiating as long as the last derivative is differentiable. Notation:
\begin{itemize}
\begin{itemize}
    \item $f''(x)$
    \item $\frac{d^2 f(x)}{dx^2}$
    \item $\frac{d^2}{dx^2} f(x)$
    \item $\frac{d^2 y}{dx^2}$ if $y = f(x)$
    \item $y''$ (sometimes)
\end{itemize}
\end{itemize}
\end{frame}
%------------------------------------------------------------------------------

%------------------------------------------------------------------------------
\begin{frame}{Topics for Tomorrow}\label{main1}

\begin{itemize}
	\begin{itemize}
		\item Increasing/Decreasing Functions
		\item Concave/Convex Functions
		\item Implicit Differentiation
		\item Partial Derivatives
		\item Taylor Series
	\end{itemize}
\end{itemize}
\end{frame}
%------------------------------------------------------------------------------

\end{document}
