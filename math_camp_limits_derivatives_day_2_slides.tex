\documentclass[aspectratio=169]{beamer}
\usepackage{booktabs}
\usepackage{graphicx}
\usepackage{adjustbox}
\newcommand{\imageframe}[1]{%
    \begin{frame}[plain]
        \begin{tikzpicture}[remember picture, overlay]
            \node[at = (current page.center), xshift = 0cm] (cover) {%
                \includegraphics[keepaspectratio, width=\paperwidth, height=\paperheight]{#1}
            };
        \end{tikzpicture}
    \end{frame}%
}
\usepackage{subfigure}
\usepackage{cancel}
\usepackage{tikz}          % For drawing figures
\usepackage{pgfplots}      % For creating the axis and plots inside tikzpicture
\pgfplotsset{compat=1.17}  % Set compatibility level for pgfplots; adjust if necessary
\usepackage{xcolor}        % For color definitions like 'alice!50!white'


% Define `accent`/`accent2` colors for theme customization
\definecolor{accent}{HTML}{006896}
\definecolor{accent2}{HTML}{E64173}

% Beamer theme
\input{preamble.tex}

% Title --------------------------------------------
\title{2025 CU Denver Math Camp - Limits \& Derivatives}
\subtitle{Day 2}
\date{August 5, 2025}
\author{Michael R. Karas}

\begin{document}

% ------------------------------------------------------------------------------
\begin{frame}
\maketitle

% \vspace{2.5mm}
{\footnotesize University of Colorado, Boulder}
\end{frame}
% ------------------------------------------------------------------------------

%------------------------------------------------------------------------------
\begin{frame}{Day 2 Topics}\label{main1}

\begin{itemize}
	\begin{itemize}
		\item Practice Problems		
		\item Increasing/Decreasing Functions
		\item Concave/Convex Functions
		\item Critical Points
		\item Partial Derivatives
		\item Taylor Series
	\end{itemize}
\end{itemize}
\end{frame}
%------------------------------------------------------------------------------

%------------------------------------------------------------------------------
\begin{frame}{Day 1 Practice Problems}\label{main1}
	\vspace{-4cm}
    \[
    f(x) = \frac{2x}{x^{2}+2}
    \]
\end{frame}
%------------------------------------------------------------------------------

%------------------------------------------------------------------------------
\begin{frame}{Day 1 Practice Problems}\label{main1}
	\vspace{-4cm}
    \[
    f(x) = e^{x^{2} + ln(x)}
    \]
\end{frame}
%------------------------------------------------------------------------------

%------------------------------------------------------------------------------
\begin{frame}{Day 1 Practice Problems}\label{main1}
	\vspace{-4cm}
    \[
    f(x) = x^{\frac{1}{4}}
    \]
    Find \( f'(x) \) and \( f''(x) \)
\end{frame}
%------------------------------------------------------------------------------

%------------------------------------------------------------------------------
\begin{frame}{Day 1 Practice Problems}\label{main1}
	\vspace{-4cm}
    \[
    f(x) = ln(x^{2} + 1)
    \]
    Find \( f'(x) \) and \( f''(x) \)
\end{frame}
%------------------------------------------------------------------------------

%------------------------------------------------------------------------------
\begin{frame}{Increasing/Decreasing Functions}\label{main1}
    Let $f$ be a differentiable function defined on an interval $[a, b]$
    \begin{itemize}
        \item $f$ is increasing on $[a, b]$ if, for every $a \leq x \leq b$, $f'(x) > 0$
        \item $f$ is decreasing on $[a, b]$ if, for every $a \leq x \leq b$, $f'(x) \leq 0$
    \end{itemize}
    Could replace $\leq$ with $<$ and say “strictly increasing” (no flat parts of $f$ on $[a, b]$), likewise with decreasing. When $f'(x) = 0$ the function is not changing, an “optimal” point.

\end{frame}
%------------------------------------------------------------------------------

%------------------------------------------------------------------------------
\begin{frame}{Increasing/Decreasing Functions Practice Problems}\label{main1}
	\vspace{-4cm}
     \[
    f(x) = x^{2}
    \]
    Find which intervals which \( f(x) \) is increasing and which intervals which \( f(x) \) is decreasing.
\end{frame}
%------------------------------------------------------------------------------

%------------------------------------------------------------------------------
\begin{frame}{Increasing/Decreasing Functions Practice Problems}\label{main1}
	\vspace{-4cm}
     \[
    f(x) = ln(x)
    \]
    Find which intervals which \( f(x) \) is increasing and which intervals which \( f(x) \) is decreasing.
\end{frame}
%------------------------------------------------------------------------------

%------------------------------------------------------------------------------
\begin{frame}{Increasing/Decreasing Functions Practice Problems}\label{main1}
	\vspace{-4cm}
     \[
    f(x) = -x^2 + 4x
    \]
    Find which intervals which \( f(x) \) is increasing and which intervals which \( f(x) \) is decreasing.
\end{frame}
%------------------------------------------------------------------------------

%------------------------------------------------------------------------------
\begin{frame}{Increasing/Decreasing Functions Practice Problems}\label{main2}
    \vspace{-4cm}
    \[
    f(x) = x^3 - 3x
    \]
    Find the intervals where \( f(x) \) is increasing and where it is decreasing.
\end{frame}
%------------------------------------------------------------------------------

%------------------------------------------------------------------------------
\begin{frame}{Concave/Convex Functions}\label{main1}
    Often it is not enough to know if a function is increasing/decreasing. We need to know its shape. Let $f$ and $I$ be as before. Then,
    \begin{itemize}
        \item $f$ is convex on $[a, b]$ if, for every $a \leq x \leq b$, $f''(x) > 0$
        \item $f$ is concave on $[a, b]$ if, for every $a \leq x \leq b$, $f''(x) < 0$
    \end{itemize}
    Convex: function is increasing at an increasing rate to the bottom of a “valley.” Concave: function is increasing at a decreasing rate to the peak of a “mountain.”
    \begin{itemize}
        \item If $f''(x) = 0$, $f$ is at an inflection point
    \end{itemize}
\end{frame}
%------------------------------------------------------------------------------

%------------------------------------------------------------------------------
\begin{frame}{Concave/Convex Functions}\label{main1}
    \begin{figure}
        \centering
        \includegraphics[width=0.9\textwidth]{/Users/michaelkaras/Library/CloudStorage/OneDrive-UCB-O365/class_materials/cu_denver_math_camp_2025/concave_convex_example.jpg}
    \end{figure}
\end{frame}
%------------------------------------------------------------------------------

%------------------------------------------------------------------------------
\begin{frame}{Concave/Convex Functions Practice Problems}\label{main1}
	\vspace{-4cm}
     \[
    f(x) = -x^2 + 4
    \]
\end{frame}
%------------------------------------------------------------------------------

%------------------------------------------------------------------------------
\begin{frame}{Concave/Convex Functions Practice Problems}\label{main2}
    \vspace{-4cm}
    \[
    f(x) = x^2 + 3x + 2
    \]
\end{frame}
%------------------------------------------------------------------------------

%------------------------------------------------------------------------------
\begin{frame}{Critical Points}\label{main1}
    One of the major uses of calculus is to find and characterize the maxima and minima of functions (such as maximizing utility and profit or minimizing costs)
    \begin{itemize}
    \item Maxima
    \begin{itemize}
        \item Local maximum at $x_0$ if $f(x) \leq f(x_0)$ for all $x$ in some open interval containing $x_0$ 
        \item Global if $f(x) \leq f(x_0)$ for all $x$ for the domain of $x_0$ 
    \end{itemize}
    \item Minima
   \begin{itemize}
        \item Local maximum at $x_0$ if $f(x) \geq f(x_0)$ for all $x$ in some open interval containing $x_0$ 
        \item Global if $f(x) \geq f(x_0)$ for all $x$ for the domain of $x_0$ 
    \end{itemize}
    \end{itemize}
\end{frame}
%------------------------------------------------------------------------------

%------------------------------------------------------------------------------
\begin{frame}{Critical Points}\label{main1}
    \begin{itemize}
    \begin{itemize}
        \item If $f'(x_0) = 0$ and $f''(x_0) < 0$, then $x_0$ is a max of $f()$
        \item If $f'(x_0) = 0$ and $f''(x_0) > 0$, then $x_0$ is a min of $f()$
        \item If $f'(x_0) = 0$ and $f''(x_0) = 0$, then $x_0$ can be a max, a min, or neither of $f()$
    \end{itemize}
    \end{itemize}
\end{frame}
%------------------------------------------------------------------------------

%------------------------------------------------------------------------------
\begin{frame}{Critical Points Example}\label{main1}
	\vspace{-4cm}
     \[
    f(x) = x^4 - 4x^3 + 4x^2 + 4
    \]
\end{frame}
%------------------------------------------------------------------------------

%------------------------------------------------------------------------------
\begin{frame}{Critical Points Example}\label{main2}
    \vspace{-4cm}
    \[
    f(x) = x^3 - 3x
    \]
\end{frame}
%------------------------------------------------------------------------------

%------------------------------------------------------------------------------
\begin{frame}{Functions of Multiple Input Variables}\label{main1}
    The function $z = f(x, y)$ takes two inputs and outputs a third.
    \begin{itemize}
        \item Functions of multiple variables assign a number from $R^n$ to a number in $R^1$, where $n$ is the number of input variables
        \item Ex. utility functions mapping the quantities of two goods to a utility value
        \begin{itemize}
        \item $u(x,y) = 4x^{\frac{1}{2}}y^{\frac{1}{2}}$
    	\end{itemize}
    	\item Graphically, these are often represented as level curves
    \end{itemize}
\end{frame}
%------------------------------------------------------------------------------

% ------------------------------------------------------------------------------------------------
\begin{frame}{Functions of Multiple Input Variables}\label{main1}
  \begin{figure}
    \caption{Indifference Curves}
    \includegraphics[width= 0.45\linewidth]{/Users/michaelkaras/Library/CloudStorage/OneDrive-UCB-O365/class_materials/cu_denver_math_camp_2025/fig3_4.JPG}
  \end{figure}
\end{frame}
% ------------------------------------------------------------------------------------------------

% ------------------------------------------------------------------------------------------------
\begin{frame}
  \vspace{-5mm}
  \begin{figure}
    \caption{Indifference Curves in 2D}
    
    \begin{tikzpicture}
      \begin{axis}[
        width = 10cm,
        height = 8cm,
        xmin = 0, xmax = 12,
        ymin = 0, ymax = 12,
        axis lines = left,
        xtick = {2, 4, 6, 8, 10}, 
        ytick = {2, 4, 6, 8, 10},
        x label style={at={(axis description cs:0.5,-0.07)},anchor=north},
        y label style={at={(axis description cs:-0.07,.5)},anchor=south},
        xlabel = {\small $x$, units of food},
        ylabel = {\small $y$, units of clothing},
        clip = false,
      ]
        % Indifference curve
        \addplot[domain = 0:12, restrict y to domain = 0:12, samples = 400, color = alice!25!white, thick]{4/x};
        \node [right,color=alice!25!white] at (axis cs: 12, 4/12) {$U = 2$};
        
        \addplot[domain = 0:12, restrict y to domain = 0:12, samples = 400, color = alice!50!white, thick]{16/x};
        \node [right,color=alice!50!white] at (axis cs: 12, 16/12) {$U = 4$};

        \addplot[domain = 0:12, restrict y to domain = 0:12, samples = 400, color = alice!75!white, thick]{36/x};
        \node [right,color=alice!75!white] at (axis cs: 12, 3) {$U = 6$};
        % the published did this wrong, but I'm keeping it to match the last figure
        
        \addplot[domain = 0:12, restrict y to domain = 0:12, samples = 400, color = alice!100!white, thick]{64/x};
        \node [right,color=alice!100!white] at (axis cs: 12, 64/12) {$U = 8$}; 
        % the published did this wrong, but I'm keeping it to match the last figure

        % Labels
        \node [anchor = north east] at (current axis.right of origin) {$x$};
        \node [above] at (current axis.above origin) {$y$};

        % Highlight Indifference Curve
        \addplot[color = black, mark = *, only marks, mark size = 2pt] 
        coordinates {(2,8) (4,4) (8,2)};

        \addplot[color = black, dotted, thick] 
          coordinates {(0, 8) (2, 8) (2, 0)};
        \node [anchor = south west] at (axis cs:2, 8) {$A$};

        \addplot[color = black, dotted, thick] 
          coordinates {(0, 4) (4, 4) (4, 0)};
        \node [anchor = south west] at (axis cs:4, 4) {$B$};

        \addplot[color = black, dotted, thick] 
          coordinates {(0, 2) (8, 2) (8, 0)};
        \node [anchor = south west] at (axis cs:8, 2) {$C$};


      \end{axis}
    \end{tikzpicture}
  \end{figure}
\end{frame}

% ------------------------------------------------------------------------------------------------

%------------------------------------------------------------------------------
\begin{frame}{Partial Derivatives}\label{main1}
    The function $z = f(x, y)$ takes two inputs and outputs a third.  We can evaluate what happens to that output when we change $x$ or $y$ (but not both).
    \begin{itemize}
        \item $\frac{\partial z}{\partial x} =$ “the derivative of $z$ with respect to $x$”
        \begin{itemize}
        \item The slope of $f$ in the cardinal direction of $x$ (e.g. “north-south”)
        \item The tangent of $f$ in the $x$ direction at a point $(x_0, y_0)$
        \end{itemize}
        \item $\frac{\partial z}{\partial y} =$ “the derivative of $z$ with respect to $y$”
        \begin{itemize}
        \item The slope of $f$ in the cardinal direction of $y$ (e.g. “east-west”)
        \item The tangent of $f$ in the $y$ direction at a point $(x_0, y_0)$
    	\end{itemize}
    \end{itemize}
\end{frame}
%------------------------------------------------------------------------------

%------------------------------------------------------------------------------
\begin{frame}{Partial Derivatives}\label{main1}
    All of these mean “take the partial derivative of $f$ with respect to $x$”:
    \begin{itemize}
        \item $f_x(x, y)$
        \item $\frac{\partial f(x, y)}{\partial x}$
        \item $\frac{\partial}{\partial x} f(x, y)$
        \item $\frac{\partial z}{\partial x}$ if $z = f(x, y)$
    \end{itemize}
\end{frame}
%------------------------------------------------------------------------------

%------------------------------------------------------------------------------
\begin{frame}{Partial Derivatives}\label{main1}
    Interpret $\frac{\partial z}{\partial x}$ as “what happens to $z$ if I change $x$, holding $y$ constant.” Likewise for $\frac{\partial z}{\partial y}$.
 	Suppose $z = x + y + xy$, find $\frac{\partial z}{\partial x}$.
    \[
    \frac{\partial z}{\partial x} = \frac{\partial}{\partial x} [x + y + xy] = \frac{\partial}{\partial x} x + \frac{\partial}{\partial x} y + \frac{\partial}{\partial x} xy = 1 + 0 + y \cdot 1 = 1 + y
    \]
\end{frame}
%------------------------------------------------------------------------------

%------------------------------------------------------------------------------
\begin{frame}{Partial Derivatives Practice Problems}
    \vspace{-4cm}
    \[
    f(x, y) = 3x + 2y
    \]
\end{frame}
%------------------------------------------------------------------------------

%------------------------------------------------------------------------------
\begin{frame}{Partial Derivatives Practice Problems}
    \vspace{-4cm}
    \[
    f(x, y) = x^2 + 4y^2 + xy
    \]
\end{frame}
%------------------------------------------------------------------------------

%------------------------------------------------------------------------------
\begin{frame}{Partial Derivatives Practice Problems}
    \vspace{-4cm}
    \[
    f(x, y) = \frac{1}{y} + x^2y - 3xy^2
    \]
\end{frame}
%------------------------------------------------------------------------------

%------------------------------------------------------------------------------
\begin{frame}{Partial Derivatives}\label{main1}
    The first partial derivatives $f_x(x, y)$ and $f_y(x, y)$ are themselves functions of $x$ and $y$. We could differentiate each of them with respect to either $x$ or $y$, so there are four second-order partial derivatives.
    \[
    f(x, y)
    \]
    \[
    \frac{\partial f}{\partial x} \quad \frac{\partial f}{\partial y}
    \]
    \[
    \frac{\partial^2 f}{\partial x^2} \quad \frac{\partial^2 f}{\partial y \partial x} \quad \frac{\partial^2 f}{\partial x \partial y} \quad \frac{\partial^2 f}{\partial y^2}
    \]

\end{frame}
%------------------------------------------------------------------------------

%------------------------------------------------------------------------------
\begin{frame}{Partial Derivatives}\label{main1}
    All of these mean “take the partial derivative of $f$ with respect to $x$ twice”:
    \begin{itemize}
        \item $f_{xx}(x, y)$
        \item $\frac{\partial^2 f(x, y)}{\partial x^2}$
        \item $\frac{\partial^2}{\partial x^2} f(x, y)$
        \item $\frac{\partial^2 z}{\partial x^2}$ if $z = f(x, y)$
    \end{itemize}
\end{frame}
%------------------------------------------------------------------------------

%------------------------------------------------------------------------------
\begin{frame}{Partial Derivatives}\label{main1}
    All of these mean “take the partial derivative of $f$ with respect to $x$, then with respect to $y$”:
    \begin{itemize}
        \item $f_{xy}(x, y)$
        \item $\frac{\partial^2 f(x, y)}{\partial y \partial x}$
        \item $\frac{\partial^2}{\partial y \partial x} f(x, y)$
        \item $\frac{\partial^2 z}{\partial y \partial x}$ if $z = f(x, y)$
    \end{itemize}
\end{frame}
%------------------------------------------------------------------------------

%------------------------------------------------------------------------------
\begin{frame}{Young's Theorem}\label{main1}
    If $f(x, y)$ is a twice differentiable function and continuous at the point $(x_0, y_0)$, then
    \[
    f_{xy}(x_0, y_0) = f_{yx}(x_0, y_0)
    \]
    The cross-partial derivatives are the same, regardless of the order in which you take them.
\end{frame}
%------------------------------------------------------------------------------

%------------------------------------------------------------------------------
\begin{frame}{Partial Derivatives Practice Problems}\label{main1}
	\vspace{-4cm}
     \[
    f(x,y) = x^{2} + y^{2} + x^{2}y^{2}
    \]
\end{frame}
%------------------------------------------------------------------------------

%------------------------------------------------------------------------------
\begin{frame}{Partial Derivatives Practice Problems}\label{main2}
    \vspace{-4cm}
    \[
    f(x,y) = xy + x^2y + xy^3
    \]
\end{frame}
%------------------------------------------------------------------------------


%------------------------------------------------------------------------------
\begin{frame}{Partial Derivatives Practice Problems}\label{main1}
	\vspace{-4cm}
     \[
    f(x,y) = x^{2}y+ln(x)y^{3}
    \]
\end{frame}
%------------------------------------------------------------------------------

%------------------------------------------------------------------------------
\begin{frame}{Taylor Series}\label{main1}
Sometimes it is too difficult (or not possible) to differentiate a function.  We can use a Taylor Series expansion to approximate the value of a function around different points.

For a function $f(x)$ with derivatives of all orders at $x=a$, the Taylor series is:
\[
f(x) = \sum_{n=0}^{\infty} \frac{f^{(n)}(a)}{n!} (x - a)^n
\]
When $a=0$, it is called the Maclaurin series.

\end{frame}
%------------------------------------------------------------------------------

%------------------------------------------------------------------------------
\begin{frame}{Taylor Series}\label{main1}
Taylor Series for $f(x) = \frac{1}{x}$
\[
f(x) = \frac{1}{x} \quad f'(x) = -\frac{1}{x^2} \quad f''(x) = \frac{2}{x^3} \quad f^{(k)}(x) = \frac{k!}{x^{k+1}}
\]
\[
P_k(x) = f(a) + f'(a)(x - a) + \frac{f''(a)}{2!}(x - a)^2 + \ldots + \frac{f^{(k)}(a)}{k!}(x - a)^k
\]
At $a=1$ of order $k==2$:
\[
P_2(x) = 1 - (x - 1) + (x - 1)^2
\]
At $a=3$ of order $k=2$:
\[
P_3(x) = \frac{1}{3} - \frac{x - 3}{3^2} + \frac{(x - 3)^2}{3^3}
\]

\end{frame}
%------------------------------------------------------------------------------

%------------------------------------------------------------------------------
\begin{frame}{Taylor Series}\label{main1}
    \begin{figure}
        \centering
        \includegraphics[width=0.6\textwidth]{/Users/michaelkaras/Library/CloudStorage/OneDrive-UCB-O365/class_materials/cu_denver_math_camp_2025/taylor_series_example.jpg}
    \end{figure}
\end{frame}
%------------------------------------------------------------------------------

%------------------------------------------------------------------------------
\begin{frame}{Taylor Series Practice Problems}\label{main1}
	\vspace{-4cm}
Write the Taylor Series Expansion of order k generated for the function \( f(x) = e^{x} \) around the point \( x = 0 \)
\end{frame}
%------------------------------------------------------------------------------



\end{document}
