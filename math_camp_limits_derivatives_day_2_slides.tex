\documentclass[aspectratio=169]{beamer}
\usepackage{booktabs}
\usepackage{graphicx}
\usepackage{adjustbox}
% \documentclass[aspectratio=43]{beamer}

% Define `accent`/`accent2` colors for theme customization
\definecolor{accent}{HTML}{006896}
\definecolor{accent2}{HTML}{E64173}

% Beamer theme
\input{preamble.tex}

% Title --------------------------------------------
\title{CU Denver Math Camp - Limits \& Derivatives}
\subtitle{Day 2}
\date{August 5, 2025}
\author{Michael R. Karas}

\begin{document}

% ------------------------------------------------------------------------------
\begin{frame}
\maketitle

% \vspace{2.5mm}
{\footnotesize University of Colorado, Boulder}
\end{frame}
% ------------------------------------------------------------------------------

%------------------------------------------------------------------------------
\begin{frame}{Day 2 Topics}\label{main1}

\begin{itemize}
	\begin{itemize}
		\item Practice Problems		
		\item Increasing/Decreasing Functions
		\item Concave/Convex Functions
		\item Implicit Differentiation
		\item Partial Derivatives
		\item Taylor Series
	\end{itemize}
\end{itemize}
\end{frame}
%------------------------------------------------------------------------------

%------------------------------------------------------------------------------
\begin{frame}{Day 1 Practice Problems}\label{main1}
	\vspace{-4cm}
    \[
    f(x) = \frac{2x}{x^{2}+2}
    \]
\end{frame}
%------------------------------------------------------------------------------

%------------------------------------------------------------------------------
\begin{frame}{Day 1 Practice Problems}\label{main1}
	\vspace{-4cm}
    \[
    f(x) = e^{x^{2} + ln(x)}
    \]
\end{frame}
%------------------------------------------------------------------------------

%------------------------------------------------------------------------------
\begin{frame}{Day 1 Practice Problems}\label{main1}
	\vspace{-4cm}
    \[
    f(x) = x(x^{2} + 1)
    \]
    Find \( f'(x) \) and \( f''(x) \)
\end{frame}
%------------------------------------------------------------------------------

%------------------------------------------------------------------------------
\begin{frame}{Day 1 Practice Problems}\label{main1}
	\vspace{-4cm}
    \[
    f(x) = ln(x^{2} + 1)
    \]
    Find \( f'(x) \) and \( f''(x) \)
\end{frame}
%------------------------------------------------------------------------------

%------------------------------------------------------------------------------
\begin{frame}{Day 1 Practice Problems}\label{main1}
	\vspace{-4cm}
    Prove that \( \frac{d}{dx}e^{x} = e^{x} \)
\end{frame}
%------------------------------------------------------------------------------

%------------------------------------------------------------------------------
\begin{frame}{Increasing/Decreasing Functions}\label{main1}
    Let $f$ be a differentiable function defined on an interval $[a, b]$
    \begin{itemize}
        \item $f$ is increasing on $[a, b]$ if, for every $a \leq x \leq b$, $f'(x) > 0$
        \item $f$ is decreasing on $[a, b]$ if, for every $a \leq x \leq b$, $f'(x) \leq 0$
    \end{itemize}
    Could replace $\leq$ with $<$ and say “strictly increasing” (no flat parts of $f$ on $[a, b]$), likewise with decreasing. When $f'(x) = 0$ the function is not changing, an “optimal” point.

\end{frame}
%------------------------------------------------------------------------------

%------------------------------------------------------------------------------
\begin{frame}{Increasing/Decreasing Functions Practice Problems}\label{main1}
	\vspace{-4cm}
     \[
    f(x) = x^{2}
    \]
    Find which intervals which \( f(x) \) is increasing and which intervals which \( f(x) \) is decreasing.
\end{frame}
%------------------------------------------------------------------------------

%------------------------------------------------------------------------------
\begin{frame}{Increasing/Decreasing Functions Practice Problems}\label{main1}
	\vspace{-4cm}
     \[
    f(x) = ln(x)
    \]
    Find which intervals which \( f(x) \) is increasing and which intervals which \( f(x) \) is decreasing.
\end{frame}
%------------------------------------------------------------------------------

%------------------------------------------------------------------------------
\begin{frame}{Increasing/Decreasing Functions Practice Problems}\label{main1}
	\vspace{-4cm}
     \[
    f(x) = e^{x}
    \]
    Find which intervals which \( f(x) \) is increasing and which intervals which \( f(x) \) is decreasing.
\end{frame}
%------------------------------------------------------------------------------

%------------------------------------------------------------------------------
\begin{frame}{Concave/Convex Functions}\label{main1}
    Often it is not enough to know if a function is increasing/decreasing. We need to know its shape. Let $f$ and $I$ be as before. Then,
    \begin{itemize}
        \item $f$ is convex on $[a, b]$ if, for every $a \leq x \leq b$, $f''(x) > 0$
        \item $f$ is concave on $[a, b]$ if, for every $a \leq x \leq b$, $f''(x) < 0$
    \end{itemize}
    Convex: function is increasing at an increasing rate to the bottom of a “valley.” Concave: function is increasing at a decreasing rate to the peak of a “mountain.”
    \begin{itemize}
        \item If $f''(x) = 0$, $f$ is at an inflection point – the acceleration switches (e.g. skiing)
    \end{itemize}
\end{frame}
%------------------------------------------------------------------------------

%------------------------------------------------------------------------------
\begin{frame}{Concave/Convex Functions}\label{main1}
    \begin{figure}
        \centering
        \includegraphics[width=0.9\textwidth]{/Users/michaelkaras/Library/CloudStorage/OneDrive-UCB-O365/class_materials/cu_denver_math_camp_2025/concave_convex_example.jpg}
    \end{figure}
\end{frame}
%------------------------------------------------------------------------------

%------------------------------------------------------------------------------
\begin{frame}{Concave/Convex Functions Practice Problems}\label{main1}
	\vspace{-4cm}
     \[
    f(x) = x - 2ln(x+1)
    \]
\end{frame}
%------------------------------------------------------------------------------

%------------------------------------------------------------------------------
\begin{frame}{Implicit Differentiation}\label{main1}
    Sometimes we can’t solve for $y = f(x)$ (or it would be very difficult) but still need to differentiate. Implicit differentiation allows us to find $f'(x)$ without a “closed-form” solution for $y = f(x)$.
    \begin{itemize}
    \begin{itemize}
        \item Differentiate both sides of the equation using all the typical rules of derivatives, treating $x$ and $y$ as variables that are both changing
        \item If you differentiate something involving $y$, multiply the answer by $y'$ (the derivative of $y = f(x)$)
        \item If you differentiate something involving $x$, do not multiply by anything new
        \item After everything is differentiated, solve for $y'$ using algebra
        \item The answer for $y'$ should be a function of $x$ (and potentially $y$)
    \end{itemize}
    \end{itemize}
\end{frame}
%------------------------------------------------------------------------------

%------------------------------------------------------------------------------
\begin{frame}{Implicit Differentiation}\label{main1}
    Suppose $x + y + xy = 5$.
    \[
    1 + 1 \cdot y' + \underbrace{1 \cdot y}_{f'(x)g(x)} + \underbrace{x \cdot 1 \cdot y'}_{f(x)g'(x)} = 0
    \]
    Using product rule:
    \[
    1 + y' + y + xy' = 0
    \]
    \[
    y'(1 + x) = -1 - y
    \]
    \[
    y' = \frac{-1 - y}{1 + x}
    \]
    The derivative is a function of both $x$ and $y$.
\end{frame}
%------------------------------------------------------------------------------

%------------------------------------------------------------------------------
\begin{frame}{Implicit Differentiation Practice Problems}\label{main1}
	\vspace{-4cm}
     \[
    x^{2} + y^{2} = 1
    \]
\end{frame}
%------------------------------------------------------------------------------

%------------------------------------------------------------------------------
\begin{frame}{Partial Derivatives}\label{main1}
    The function $z = f(x, y)$ takes two inputs and outputs a third.  We can evaluate what happens to that output when we change $x$ or $y$ (but not both).
    \begin{itemize}
        \item $\frac{\partial z}{\partial x} =$ “the derivative of $z$ with respect to $x$”
        \begin{itemize}
        \item The slope of $f$ in the cardinal direction of $x$ (e.g. “north-south”)
        \item The tangent of $f$ in the $x$ direction at a point $(x_0, y_0)$
        \end{itemize}
        \item $\frac{\partial z}{\partial y} =$ “the derivative of $z$ with respect to $y$”
        \begin{itemize}
        \item The slope of $f$ in the cardinal direction of $y$ (e.g. “east-west”)
        \item The tangent of $f$ in the $y$ direction at a point $(x_0, y_0)$
    	\end{itemize}
    \end{itemize}
\end{frame}
%------------------------------------------------------------------------------

%------------------------------------------------------------------------------
\begin{frame}{Partial Derivatives}\label{main1}
    All of these mean “take the partial derivative of $f$ with respect to $x$”:
    \begin{itemize}
        \item $f_x(x, y)$
        \item $\frac{\partial f(x, y)}{\partial x}$
        \item $\frac{\partial}{\partial x} f(x, y)$
        \item $\frac{\partial z}{\partial x}$ if $z = f(x, y)$
    \end{itemize}
\end{frame}
%------------------------------------------------------------------------------

%------------------------------------------------------------------------------
\begin{frame}{Partial Derivatives}\label{main1}
    Interpret $\frac{\partial z}{\partial x}$ as “what happens to $z$ if I change $x$, holding $y$ constant.” Likewise for $\frac{\partial z}{\partial y}$.
 	Suppose $z = x + y + xy$, find $\frac{\partial z}{\partial x}$.
    \[
    \frac{\partial z}{\partial x} = \frac{\partial}{\partial x} [x + y + xy] = \frac{\partial}{\partial x} x + \frac{\partial}{\partial x} y + \frac{\partial}{\partial x} xy = 1 + 0 + y \cdot 1 = 1 + y
    \]
\end{frame}
%------------------------------------------------------------------------------

%------------------------------------------------------------------------------
\begin{frame}{Partial Derivatives}\label{main1}
    The first partial derivatives $f_x(x, y)$ and $f_y(x, y)$ are themselves functions of $x$ and $y$. We could differentiate each of them with respect to either $x$ or $y$, so there are four second-order partial derivatives.
    \[
    f(x, y)
    \]
    \[
    \frac{\partial f}{\partial x} \quad \frac{\partial f}{\partial y}
    \]
    \[
    \frac{\partial^2 f}{\partial x^2} \quad \frac{\partial^2 f}{\partial y \partial x} \quad \frac{\partial^2 f}{\partial x \partial y} \quad \frac{\partial^2 f}{\partial y^2}
    \]

\end{frame}
%------------------------------------------------------------------------------

%------------------------------------------------------------------------------
\begin{frame}{Partial Derivatives}\label{main1}
    All of these mean “take the partial derivative of $f$ with respect to $x$ twice”:
    \begin{itemize}
        \item $f_{xx}(x, y)$
        \item $\frac{\partial^2 f(x, y)}{\partial x^2}$
        \item $\frac{\partial^2}{\partial x^2} f(x, y)$
        \item $\frac{\partial^2 z}{\partial x^2}$ if $z = f(x, y)$
    \end{itemize}
\end{frame}
%------------------------------------------------------------------------------

%------------------------------------------------------------------------------
\begin{frame}{Partial Derivatives}\label{main1}
    All of these mean “take the partial derivative of $f$ with respect to $x$, then with respect to $y$”:
    \begin{itemize}
        \item $f_{xy}(x, y)$
        \item $\frac{\partial^2 f(x, y)}{\partial y \partial x}$
        \item $\frac{\partial^2}{\partial y \partial x} f(x, y)$
        \item $\frac{\partial^2 z}{\partial y \partial x}$ if $z = f(x, y)$
    \end{itemize}
\end{frame}
%------------------------------------------------------------------------------

%------------------------------------------------------------------------------
\begin{frame}{Young's Theorem}\label{main1}
    If $f(x, y)$ is a twice differentiable function and continuous at the point $(x_0, y_0)$, then
    \[
    f_{xy}(x_0, y_0) = f_{yx}(x_0, y_0)
    \]
    The cross-partial derivatives are the same, regardless of the order in which you take them.
\end{frame}
%------------------------------------------------------------------------------

%------------------------------------------------------------------------------
\begin{frame}{Partial Derivatives Practice Problems}\label{main1}
	\vspace{-4cm}
     \[
    f(x,y) = x^{2} + y^{2} + x^{2}y^{2}
    \]
\end{frame}
%------------------------------------------------------------------------------

%------------------------------------------------------------------------------
\begin{frame}{Partial Derivatives Practice Problems}\label{main1}
	\vspace{-4cm}
     \[
    f(x,y) = x^{2}y+ln(x)y^{3}
    \]
\end{frame}
%------------------------------------------------------------------------------

%------------------------------------------------------------------------------
\begin{frame}{Taylor Series}\label{main1}
Sometimes it is too difficult (or not possible) to differentiate a function.  We can use a Taylor Series expansion to approximate the value of a function around different points.

For a function $f(x)$ with derivatives of all orders at $x=a$, the Taylor series is:
\[
f(x) = \sum_{n=0}^{\infty} \frac{f^{(n)}(a)}{n!} (x - a)^n
\]
When $a=0$, it is called the Maclaurin series.

\end{frame}
%------------------------------------------------------------------------------

%------------------------------------------------------------------------------
\begin{frame}{Taylor Series}\label{main1}
Taylor Series for $f(x) = \frac{1}{x}$
\[
f(x) = \frac{1}{x} \quad f'(x) = -\frac{1}{x^2} \quad f''(x) = \frac{2}{x^3} \quad f^{(k)}(x) = \frac{k!}{x^{k+1}}
\]
\[
P_k(x) = f(a) + f'(a)(x - a) + \frac{f''(a)}{2!}(x - a)^2 + \ldots + \frac{f^{(k)}(a)}{k!}(x - a)^k
\]
At $a=1$ of order $k==2$:
\[
P_2(x) = 1 - (x - 1) + (x - 1)^2
\]
At $a=3$ of order $k=2$:
\[
P_3(x) = \frac{1}{3} - \frac{x - 3}{3^2} + \frac{(x - 3)^2}{3^3}
\]

\end{frame}
%------------------------------------------------------------------------------

%------------------------------------------------------------------------------
\begin{frame}{Taylor Series}\label{main1}
    \begin{figure}
        \centering
        \includegraphics[width=0.6\textwidth]{/Users/michaelkaras/Library/CloudStorage/OneDrive-UCB-O365/class_materials/cu_denver_math_camp_2025/taylor_series_example.jpg}
    \end{figure}
\end{frame}
%------------------------------------------------------------------------------

%------------------------------------------------------------------------------
\begin{frame}{Taylor Series Practice Problems}\label{main1}
	\vspace{-4cm}
Write the Taylor Series Expansion of order k generated for the function \( f(x) = e^{x} \) around the point \( x = 0 \)
\end{frame}
%------------------------------------------------------------------------------



\end{document}
